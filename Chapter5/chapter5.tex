\chapter{Conclusion and Future Work}\label{chap:conclusion}
\section{Conclusion}
In this thesis, we perform the non-local analysis of natural images and propose a non-local modeling method for no-reference image quality assessment. The conclusion is summarized as three folds.

Firstly, the local modeling method CNN and non-local modeling on natural images are separately analyzed and discussed. The local modeling encodes spatially proximate local neighborhoods. Evidence has shown that the local modeling features derived by CNN are effective and robust. Nevertheless, its generalized representation capability is curbed and harnessed by the shortcomings of the locality inductive bias, such as local feature extraction, equally processed content, and failure to model geometric and relational causal dependencies. On the other hand, the non-local analysis of natural images is introduced from the perspectives of non-local statistics, non-local dependency and relational modeling, and effective image-specific prior, revealing the superiority of applying the non-local modeling for visual quality assessment.

Secondly, in the view that HVS perceives image quality with long-range dependencies constructed among different regions, we propose a non-local behavior, named NLNet, for perceptual quality assessment. The non-local modeling establishes the spatial integration of information by long- and short-range communications with different spatial weighting functions. Specifically, the non-local features and long-range dependencies are learned and comprehended via a two-stage superpixel-based graph neural network approach. Extensive experimental results on the individual distortion type demonstrate that the non-local modeling method manages to handle a wide variety of global distortions,~\ie, the globally and uniformly distributed distortions with non-local recurrences. Meanwhile, it also maintains sensitivity to local distortions,~\ie, the local nonuniform-distributed distortions in a local region. In addition, it holds considerable prediction power in assessing the quality of noisy and compressed images. Lastly, the outstanding performance in the cross-dataset setting reveals the high generalization capability of our method, shedding light on the exploration of the generalized NR-IQA models.

Thirdly, the ablation study has shown that the non-local modeling is complementary to traditional local methods. Both local and non-local features contribute to image quality assessment. The unilateral local features or non-local features are not truly adequate to evaluate perceptual quality. The combination of the two is superior for visual quality assessment. Thus, this thesis opens the door to considering the non-local concept in the field, together with the local features from the local modeling methods.

\section{Potential Future Work}
In the future, several intriguing research questions are waiting to answer. In this thesis, we propose four research topics as potential future work.

\paragraph{No-reference Image Quality Assessment and Image Processing Systems Optimization via Non-local Statistics}~The alterations of non-local statistics between the reference and distorted images reflect the image degradation process. Robust IQA models with high generalization capability would be built by measuring non-local statistics from images. In the meantime, algorithms can also benefit from local-based statistics, such as BRISQUE~\citep{mittal2012no}. By complementing each other's shortcomings, advanced visual quality assessment methods would emerge. Thus, we are excited to further explore the non-local-guided natural scene statistics during the image degradation process of synthetic images. In addition, non-local statistics could also be utilized to estimate the optimal parameters of image processing systems,~\eg, noise variance~\citep{mittal2012no, zhu2010automatic}.

\paragraph{Blindly Assess Image Quality in the Wild via Non-local Modeling}~In this thesis, we propose a non-local modeling method for synthetically distorted natural images, where most distortions are globally and uniformly distributed with non-local recurrences. However, in real-life scenarios, images are usually authentically contaminated with more distortion diversities (\ie, complex, various, and mixed distortions), larger content variations, and their quality would be spatially different due to local distortions. In order to solve the real-world blind IQA problem, we are curious about the impacts of the non-local modeling on authentically distorted images.

\paragraph{Image Quality Assessment based on the Surround Modulation Mechanism in HVS}~In HVS, firing patterns of the CRF can be modulated,~\ie, inhibited (suppressed) or excitatory (facilitated) by the nCRF. The local response is modulated and affected by the outside-located stimuli, and the surrounding visual context helps to predict the local stimuli's content. The Surround Modulation (Contextual Modulation) mechanism catches the non-local and long-range spatial interactions between the CRF and its surroundings. Thus, the non-local modeling methods would be explored to model and validate the Surround Modulation mechanism, long-range interactions, and information integration between the CRF and the nCRF inside HVS.

\paragraph{Quality Evaluation, Analyses, and Processing of Artificial Intelligence-generated Content}~Nowadays, Artificial Intelligence-generated Content (AI-Generated Content or AIGC) gains much more popularity in academia and industry. In the next decade, it can be expected that a growing number of content in the Internet will be produced by AI. Thus, quality assessment will play a indispensable and fundamental role in measuring visual quality of AIGC~\citep{zhang2023perceptual}. In addition, novel methods, techniques, and toolkits will be required to analyze, process, and improve these content~\citep{wu2023ai}.